\section{Components}
In this section we will explain all the components
used during the building of the robot and how to get them.

\subsection{Mechanical components}
This components do not have any electronics and the important
thing about them is their mechanical function.
\subsubsection{Central Body}

Used to host the flywheel, a motor and a bearing. It also has lateral
tabs so it can easly be joined with the lateral body part.
It's 3D printed and in the same orientation you see in figure
\ref{fig: central body}
\begin{figure}[H]
    \centering
    \includegraphics[width=10cm]{img/components/central_body.png}
    \caption{3D Model of the central body.}
    \label{fig: central body}
\end{figure}

\subsubsection{Lateral Body}

Used to host the the wheel motor and the rest of the electronic components. 
It also has lateral tabs so it can easly be joined with the central body part.
It's 3D printed and in the same orientation you see in figure
\ref{fig: lateral body}
\begin{figure}[H]
    \centering
    \includegraphics[width=10cm]{img/components/lateral_body.png}
    \caption{3D Model of lateral body.}
    \label{fig: lateral body}
\end{figure}
\subsubsection{Wheels}
The wheels are 3D printed. The orientation we used to print them
is putting the motor axis vertical. Each wheel is produced by two
symetric parts that are glued toghether arround the motor.
\begin{figure}[H]
    \centering
    \includegraphics[width=10cm]{img/components/wheel.png}
    \caption{3D Model of a wheel.}
    \label{fig:}
\end{figure}
\subsubsection{Flywheel}
\subsubsection{Bearings}
\subsection{Electronic components}
\subsubsection{Raspberry Pi}
The Raspberry Pi is a small single-board computer. We are using Raspberry Pi 3 Model B. It has GPIO pins.
\begin{figure}[H]
    \centering
    \includegraphics[width=10cm]{img/components/raspberry_pi.png}
    \caption{Raspberry Pi picture}
    \label{fig:}
\end{figure}
\begin{center}
    \begin{tabular}{ |c|c| }
        \hline
        Weight          & 42 g      \\
        \hline
        Price per unit  & 35 euros \\
        \hline
        Number of units & 1        \\
        \hline
    \end{tabular}
\end{center}
\subsubsection{Batteries}
Provide power to our motors and to the Raspberry Pi.
\subsubsection{Breadboard}
A breadboard is a construction base for prototyping of electronics.
\subsubsection{DC Motor}
A DC motor is a class of rotary electrical machines that converts
direct current electrical energy into mechanical energy. The most common
types rely on the forces produced by magnetic fields.
\begin{center}
    \begin{tabular}{ |c|c| }
        \hline
        Operating voltage       & between 3 V and 9 V \\
        \hline
        Free-run speed at 6 V   & 176 RPM             \\
        \hline
        Free-run current at 6 V & 80 mA               \\
        \hline
        Stall current at 6V     & 900 mA              \\
        \hline
        Stall current at 6V     & 5 kg·cm             \\
        \hline
        Gear ratio              & 1:35                \\
        \hline
        Reductor size           & 21 mm               \\
        \hline
        Weight                  & 85 g                \\
        \hline
        Price per unit          & 10 euros            \\
        \hline
        Number of units         & 3                   \\
        \hline
    \end{tabular}
\end{center}

\subsubsection{H Bridge}
An H bridge is an electronic circuit that switches the polarity of a
voltage applied to a load. These circuits are often used in robotics
and other applications to allow DC motors to run forwards or backwards.
\subsubsection{Rotatory Encoder}
A rotary encoder, also called a shaft encoder, is an electro-mechanical device that converts the
angular position or motion of a shaft or axle to analog or digital output signals.
\subsubsection{Accelerometer}
An accelerometer is a device that measures proper acceleration. Proper acceleration, being the
acceleration (or rate of change of velocity) of a body in its own instantaneous rest frame, is not the same as coordinate acceleration, being the acceleration in a fixed coordinate system.
For example, an accelerometer at rest on the surface of the Earth will measure an acceleration due
to Earth's gravity, straight upwards (by definition) of g ≈ 9.81 m/s2.
