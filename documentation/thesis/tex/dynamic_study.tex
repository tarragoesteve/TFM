\section{Dynamics}

In order to study the dynamics of the robot we will use the Lagrange mechanics.

\subsection{Rectilinear movement with $r_{flywheel}$ fixed to $r_{flywheel-min}$}
To reduce the number of variables we wil study the case of rectilinear
movement by imposing that both wheels turn at the same speed.

The generalized coordinates ($q$) will be:
\begin{enumerate}
	\item $\phi_{ground-wheel}$: rotation of the wheel respect the ground.
	\item $\phi_{wheel-platform}$: rotation of the platform respect the wheel.
	\item $\phi_{platform-flywheel}$: rotation of the platform respect the wheel.
\end{enumerate}

We will use two auxiliary variables:
\begin{enumerate}
	\item $\phi_{ground-platform}=\phi_{ground-wheel}+\phi_{wheel-platform}$: rotation of the platform respect the ground.
	\item $\phi_{ground-flywheel}=\phi_{ground-platform}+\phi_{platform-flywheel}$: rotation of the flywheel respect the ground.
\end{enumerate}

The total potential energy:
\begin{equation}
	V = m_{flywheel}\cdot (r_{flywheel-max}-r_{flywheel-min}) \cdot sin(\phi_{ground-flywheel}) \cdot g	
\end{equation}


The total kinetic energy:
\begin{equation}
	T = \frac{1}{2}\cdot[\dot{\phi}_{ground-wheel}^2\cdot I_{wheel}
	+ \dot{\phi}_{ground-platform}^2 \cdot I_{platform}
	+ \dot{\phi}_{ground-flywheel}^2\cdot I_{flywheel}
	+ \dot{\phi}_{ground-wheel}^2\cdot r_{wheel}^2\cdot m_{total}]	
\end{equation}

The Lagrangian is defined as:
\begin{equation}
	L=T-V	
\end{equation}

Lagrange's equation is:
\begin{equation}
	\frac{d}{dt}(\frac{\partial L}{\partial \dot{q}_j})=
	\frac{\partial L}{\partial q_j}	
\end{equation}

So in our case:
\begin{equation}
	\frac{\partial L}{\partial \dot{\phi}_{ground-wheel}}=
	\dot{\phi}_{ground-wheel} \cdot I_{wheel}
	+ \dot{\phi}_{ground-platform} \cdot I_{platform}
	+ \dot{\phi}_{ground-flywheel}\cdot I_{flywheel}
	+ \dot{\phi}_{ground-wheel}\cdot r_{wheel}^2\cdot m_{total}
\end{equation}

\begin{equation}
	\frac{\partial L}{\partial \dot{\phi}_{wheel-platform}}=
	\dot{\phi}_{ground-platform} \cdot I_{platform}
	+ \dot{\phi}_{ground-flywheel}\cdot I_{flywheel}
\end{equation}

\begin{equation}
	\frac{\partial L}{\partial \dot{\phi}_{wheel-platform}}=
	\dot{\phi}_{platform-flywheel}\cdot I_{flywheel}
\end{equation}

We define M as the following matrix:
\begin{equation}
	\begin{pmatrix} 
		I_{wheel} + I_{platform} + I_{flywheel} + r_{wheel}^2 \cdot m_{total} &
		I_{platform} + I_{flywheel} &
		I_{flywheel}\\
		I_{platform} + I_{flywheel} &
		I_{platform} + I_{flywheel} &
		I_{flywheel}\\
		I_{flywheel} &
		I_{flywheel} &
		I_{flywheel}\\
		\end{pmatrix}
\end{equation}

In matrix form and with our generalized coordinates:
\begin{equation}
	\frac{\partial L}{\partial \dot{q}} =
	M \cdot \dot{q}
\end{equation}

\begin{equation}
	\frac{d}{dt}(\frac{\partial L}{\partial \dot{q}}) = 	M \cdot \ddot{q}	
\end{equation}

Let a as a constant:
\begin{equation}
	a = - m_{flywheel}\cdot (r_{flywheel-max}-r_{flywheel-min}) \cdot g	
\end{equation}

\begin{equation}
	\frac{\partial L}{\partial q} = a \cdot cos(\phi_{ground-flywheel}) \cdot 	
	\begin{pmatrix}
	1 \\ 1 \\ 1
	\end{pmatrix}	
\end{equation}

So using Lagrange's equation we get:
\begin{equation}
	\boxed{
	M \cdot \ddot{q} = a \cdot cos(\phi_{ground-flywheel}) \cdot 	
	\begin{pmatrix}
	1 \\ 1 \\ 1
	\end{pmatrix}
	}
\end{equation}