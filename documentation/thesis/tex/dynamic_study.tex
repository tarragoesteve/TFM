\section{Dynamics}

In order to study the dynamics of the robot we will use Lagrange mechanics.

\subsection{Rectilinear movement with $r_{flywheel}$ fixed to $r_{flywheel-min}$}
To reduce the number of variables we wil study the case of rectilinear
movement by imposing that both wheels turn at the same speed. We will also set
the radius of the free weight to $r_{flywheel-min}$.

The generalized coordinates ($q$) will be:
\begin{enumerate}
	\item $\phi_{ground-wheel}$: rotation of the wheel respect the ground.
	\item $\phi_{wheel-platform}$: rotation of the platform respect the wheel.
	\item $\phi_{platform-flywheel}$: rotation of the flywheel respect the platform.
\end{enumerate}

We will use two auxiliary variables:
\begin{enumerate}
	\item $\phi_{ground-platform}=\phi_{ground-wheel}+\phi_{wheel-platform}$: rotation of the platform respect the ground.
	\item $\phi_{ground-flywheel}=\phi_{ground-platform}+\phi_{platform-flywheel}$: rotation of the flywheel respect the ground.
\end{enumerate}

The total potential energy:
\begin{equation}
	V = m_{cylinder}\cdot (r_{flywheel-min}-r_{flywheel-max}) \cdot cos(\phi_{ground-flywheel}) \cdot g	
\end{equation}


The total kinetic energy:
\begin{equation}
	T = \frac{1}{2}\cdot[\dot{\phi}_{ground-wheel}^2\cdot I_{wheel}
	+ \dot{\phi}_{ground-platform}^2 \cdot I_{platform}
	+ \dot{\phi}_{ground-flywheel}^2\cdot I_{flywheel}
	+ \dot{\phi}_{ground-wheel}^2\cdot r_{wheel}^2\cdot m_{total}]	
\end{equation}

The Lagrangian is defined as:
\begin{equation}
	L=T-V	
\end{equation}

Lagrange's equation is:
\begin{equation}
	\frac{d}{dt}(\frac{\partial L}{\partial \dot{q}_j})=
	\frac{\partial L}{\partial q_j}	+ F_{j}
\end{equation}

So in our case:
\begin{equation}
	\frac{\partial L}{\partial \dot{\phi}_{ground-wheel}}=
	\dot{\phi}_{ground-wheel} \cdot I_{wheel}
	+ \dot{\phi}_{ground-platform} \cdot I_{platform}
	+ \dot{\phi}_{ground-flywheel}\cdot I_{flywheel}
	+ \dot{\phi}_{ground-wheel}\cdot r_{wheel}^2\cdot m_{total}
\end{equation}

\begin{equation}
	\frac{\partial L}{\partial \dot{\phi}_{wheel-platform}}=
	\dot{\phi}_{ground-platform} \cdot I_{platform}
	+ \dot{\phi}_{ground-flywheel}\cdot I_{flywheel}
\end{equation}

\begin{equation}
	\frac{\partial L}{\partial \dot{\phi}_{platform-flywheel}}=
	\dot{\phi}_{ground-flywheel}\cdot I_{flywheel}
\end{equation}

We define M as the following matrix:
\begin{equation}
	\begin{pmatrix} 
		I_{wheel} + I_{platform} + I_{flywheel} + r_{wheel}^2 \cdot m_{total} &
		I_{platform} + I_{flywheel} &
		I_{flywheel}\\
		I_{platform} + I_{flywheel} &
		I_{platform} + I_{flywheel} &
		I_{flywheel}\\
		I_{flywheel} &
		I_{flywheel} &
		I_{flywheel}\\
		\end{pmatrix}
\end{equation}

In matrix form and with our generalized coordinates:
\begin{equation}
	\frac{\partial L}{\partial \dot{q}} =
	M \cdot \dot{q}
\end{equation}

\begin{equation}
	\frac{d}{dt}(\frac{\partial L}{\partial \dot{q}}) = 	M \cdot \ddot{q}	
\end{equation}

Let a be a constant:
\begin{equation}
	a = m_{cylinder}\cdot (r_{flywheel-min}-r_{flywheel-max}) \cdot g	
\end{equation}

\begin{equation}
	\frac{\partial L}{\partial q} = a \cdot sin(\phi_{ground-flywheel}) \cdot 	
	\begin{pmatrix}
	1 \\ 1 \\ 1
	\end{pmatrix}	
\end{equation}

So using Lagrange's equation we get:
\begin{equation}
	\boxed{
	M \cdot \ddot{q} = a \cdot sin(\phi_{ground-flywheel}) \cdot 	
	\begin{pmatrix}
	1 \\ 1 \\ 1
	\end{pmatrix} + F
	}
\end{equation}

\subsection{Rectilinear movement with $r_{flywheel}$ free}

The generalized coordinates ($q$) will be:
\begin{enumerate}
	\item $\phi_{ground-wheel}$: rotation of the wheel respect the ground.
	\item $\phi_{wheel-platform}$: rotation of the platform respect the wheel.
	\item $\phi_{platform-flywheel}$: rotation of the flywheel respect the platform.
	\item $r$: distance from the center of the flywheel of the free cylinder.

\end{enumerate}

We will use two auxiliary variables:
\begin{enumerate}
	\item $\phi_{ground-platform}=\phi_{ground-wheel}+\phi_{wheel-platform}$: rotation of the platform respect the ground.
	\item $\phi_{ground-flywheel}=\phi_{ground-platform}+\phi_{platform-flywheel}$: rotation of the flywheel respect the ground.
\end{enumerate}

The total potential energy:
\begin{equation}
	V = m_{cylinder}\cdot (r-r_{flywheel-max}) \cdot cos(\phi_{ground-flywheel}) \cdot g	
\end{equation}


The total kinetic energy:
\begin{multline}
	T = \frac{1}{2}\cdot[\dot{\phi}_{ground-wheel}^2\cdot I_{wheel}
	+ \dot{\phi}_{ground-platform}^2 \cdot I_{platform}
	+ \dot{\phi}_{ground-flywheel}^2\cdot I_{flywheel}(r)\\
	+ \dot{\phi}_{ground-wheel}^2\cdot r_{wheel}^2\cdot m_{total}
	+ \dot{r}^2\cdot m_{cylinder}]	
\end{multline}

The Lagrangian is defined as:
\begin{equation}
	L=T-V	
\end{equation}

Lagrange's equation is:
\begin{equation}
	\frac{d}{dt}(\frac{\partial L}{\partial \dot{q}_j})=
	\frac{\partial L}{\partial q_j}	+ F_{j}
\end{equation}

So in our case:
\begin{multline}
	\frac{\partial L}{\partial \dot{\phi}_{ground-wheel}}=
	\dot{\phi}_{ground-wheel} \cdot I_{wheel}
	+ \dot{\phi}_{ground-platform} \cdot I_{platform}\\
	+ \dot{\phi}_{ground-flywheel}\cdot I_{flywheel}(r)
	+ \dot{\phi}_{ground-wheel}\cdot r_{wheel}^2\cdot m_{total}
\end{multline}

\begin{equation}
	\frac{\partial L}{\partial \dot{\phi}_{wheel-platform}}=
	\dot{\phi}_{ground-platform} \cdot I_{platform}
	+ \dot{\phi}_{ground-flywheel}\cdot I_{flywheel}(r)
\end{equation}

\begin{equation}
	\frac{\partial L}{\partial \dot{\phi}_{platform-flywheel}}=
	\dot{\phi}_{ground-flywheel}\cdot I_{flywheel}(r)
\end{equation}

\begin{equation}
	\frac{\partial L}{\partial \dot{r}}=
	\dot{r}\cdot m_{cylinder}
\end{equation}

We define M as the following matrix:
\begin{equation}
	\begin{pmatrix} 
		I_{wheel} + I_{platform} + I_{flywheel}(r) + r_{wheel}^2 \cdot m_{total} &
		I_{platform} + I_{flywheel}(r) &
		I_{flywheel}(r)&
		0\\
		I_{platform} + I_{flywheel}(r) &
		I_{platform} + I_{flywheel}(r) &
		I_{flywheel}(r)&
		0\\
		I_{flywheel}(r) &
		I_{flywheel}(r) &
		I_{flywheel}(r) &
		0\\
		0 &
		0 &
		0 &
		m_{cylinder}\\
		\end{pmatrix}
\end{equation}

In matrix form and with our generalized coordinates:
\begin{equation}
	\frac{\partial L}{\partial \dot{q}} =
	M \cdot \dot{q}
\end{equation}

\begin{equation}
	\frac{d}{dt}(\frac{\partial L}{\partial \dot{q}}) =
	 	M \cdot \ddot{q} + \dot{M} \cdot \dot{q} 	
\end{equation}

Let's recall the definition of $I_{flywheel}(r)$
\begin{equation}
	I_{flywheel}(r)= m_{cylinder} \cdot r^2 + C	
\end{equation}


Compute the derivative
\begin{equation}
	\dot{I}_{flywheel}(r)= 2 \cdot m_{cylinder} \cdot r \cdot \dot{r} 	
\end{equation}

We compute $\dot{M}$ as the following matrix:
\begin{equation}
	\dot{M}=
	\dot{I}_{flywheel}(r) \cdot
	\begin{pmatrix} 
		1&
		1&
		1&
		0\\
		1 &
		1 &
		1&
		0\\
		1 &
		1 &
		1 &
		0\\
		0 &
		0 &
		0 &
		0\\
		\end{pmatrix}
\end{equation}

Let $a$ be::
\begin{equation}
	\frac{\partial L}{\partial q_{1..3}} = a = m_{cylinder}\cdot (r-r_{flywheel-max}) \cdot g \cdot sin(\phi_{ground-flywheel})	
\end{equation}

Let $b$ be:
\begin{equation}
	\frac{\partial L}{\partial r} = b = 2 \cdot m_{cylinder} \cdot r  \cdot \dot{\phi}_{ground-flywheel}^2  - m_{cylinder}\cdot g \cdot cos(\phi_{ground-flywheel})	
\end{equation}

\begin{equation}
	\frac{\partial L}{\partial q} = \begin{pmatrix}
	a \\ a \\ a \\ b
	\end{pmatrix}	
\end{equation}

So using Lagrange's equation we get:
\begin{equation}
	\boxed{
	M \cdot \ddot{q} + \dot{M} \cdot \dot{q}= 
	\begin{pmatrix}
		a \\ a \\ a \\ b
		\end{pmatrix}	
 	+ F
	}
\end{equation}