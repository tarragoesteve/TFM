\section{Conclusion}
Thinking, designing and building a prototype is a big and rewarding challenge.
There are many things that do not work as expected and you need to iterate again
and find a solution to the problem. Some examples are too much deformation in the platform,
too noisy speed measurements or 3D printed parts that we couldn't unpack. You may see a gallery of photos
of the process in this \href{link}{https://photos.app.goo.gl/dJozKgFnNSdTapSy6} (\href{https://photos.app.goo.gl/dJozKgFnNSdTapSy6}{https://photos.app.goo.gl/dJozKgFnNSdTapSy6}).

On the other hand we came out with a model using the fundamental mechanics equations, but we are missing
some frictions and other forces that do not appear in our model. This made our robot behave a little bit
different from expected from simulations. With the flywheel mode the robot can hardly start moving and with
the pendulum mode the platform needs to turn a little to start and brake.

For a possible following iteration I would recommend much more powerful motors and a more robust structure.
The robot now is not behaving perfectly but can move around more or less as expected, as you can see in this
\href{https://youtu.be/yWOlyIL4cP0}{video} (\href{https://youtu.be/yWOlyIL4cP0}{https://youtu.be/yWOlyIL4cP0}).