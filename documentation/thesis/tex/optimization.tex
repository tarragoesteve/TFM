\section{Optimization Setup}
In this section we will set up the requirements and the cost function we want to optimize.

\subsection{Restrictions}
The restrictions are a list of inequalities that our system has to fulfill.
The first restriction is due to the initial design requirements. Finally, the other three are 
somehow arbitrary but will help us to reduce the size of the robot.
\begin{enumerate}
\item We will place our flywheel in a hole on our robot. We don't want to touch the ground
in any configuration so:
\begin{figure}[ht]
	\centering
	\includegraphics[width=7cm]{img/flywheel_hole.jpg}
	\caption{Flywheel hole diagram}
	\label{fig:Flywheel hole diagram}
\end{figure}
\[r_{wheel}> \sqrt{(r_{flywheel} + b)^2+(\frac{h}{2})^2}\]
\item Being able  to insert the robot in to a wheel of diameter 0.5 m so:
\begin{figure}[ht]
	\centering
	\includegraphics[width=7cm]{img/external_diameter.jpg}
	\caption{External diameter diagram}
	\label{fig:External diameter diagram}
\end{figure}
\[0.25 m > \sqrt{r_{wheel}^2 + L^2/4}\]
\item We can place all electronic the devices:
\[L > 0.3m + w \]
\item Maximum weight of the robot: 5 kg
\end{enumerate}

\subsection{Requirements}
We would like our robot to reach some mechanical specifications. These are related to mechanical equations
that we will develop in the next section. They refer to the max speed, acceleration and terrain inclination the robot may achieve
while controlling its platform inclination. We have divided our specifications in two blocks
according to the two mechanisms.

\textbf{Flywheel mode}
\begin{enumerate}
	\item $\dot{y}_{max}$ (equation \ref{Maximum speed flywheel}) $> 0.1m/s$.
	\item $\ddot{y}_{max}$ (equation \ref{maximum acceleration flywheel}) $> 1m/s^2$.
	\item $sin(\alpha_{max})$ (equation \ref{Maximum angle using flywheel system}) $>0.16$.
\end{enumerate}

\textbf{Pendulum mode}
\begin{enumerate}
	\item $\dot{y}_{max}$ (equation \ref{maximum speed pendulum}) $>1m/s$.
	\item $\ddot{y}_{max}$ (equation \ref{maximum acceleration pendulum}) $>0.1m/s^2$.
	\item $sin(\alpha_{max})$ (equation \ref{Maximum angle using pendulum system}) $> 0.02$.
\end{enumerate}
	


\subsection{Cost function}
In addition to fulfilling the previous inequalities, we will adjust our design parameters [w (width of the
cylinders), N(number of cylinders), r wheel and r flywheel] to
minimize a cost function. 

We will maximize the maximum sinus in the pendulum mode 
(equation \ref{Maximum angle using pendulum system}) because
it gives the robot the capacity to deliver force in a permanent state.

And we will also maximize the square of the max speed the robot can achieve
in flywheel mode (equation \ref{Maximum speed flywheel}) because it is
proportional to the energy the robot can deliver using the flywheel at a certain moment.

Both equations try to maximize different modes so the robot we have a compromise between the two of them. 

\begin{equation}
	cost(r_{flywheel},r_{wheel},w,N) = - sin(\alpha_{max})_{pendulum} -\dot{y}^2_{max-flywheel}
	\label{eq: cost}
\end{equation}
In the next section we will find out what are the values 
of these equations with the construction parameters. 
%For reference these terms are equal to:
%\begin{equation*}
%	m_{cylinder} = \rho * w * \pi * (\frac{r_{flywheel}}{3})^2
%\end{equation*}
%\begin{equation*}
%	sin(\alpha_{max})_{pendulum} = \frac{m_{cylinder} \cdot  (r_{max} - r_{min})}{m_{total} \cdot r_{wheel}} = \frac{m_{cylinder} \cdot  (\frac{r_{flywheel}}{3})}{(m_{rest} + N \cdot m_{cylinder})\cdot r_{wheel}} 	
%\end{equation*}
%\begin{equation*}
%	\dot{y}_{max} = r_{wheel} \cdot  R \cdot  \dot{\theta}_{max} =r_{wheel} \cdot  \frac{ N \cdot  m_{cylinder} \cdot  (\frac{2\cdot r_{flywheel}}{3})^2}
%    {r_{wheel}^2\cdot (m_{rest} + N \cdot m_{cylinder}) +  2\cdot I_{wheel}} \cdot  \dot{\theta}_{max}
%\end{equation*}
